\chapter{Introduzione}
\label{chap:intro}
\newcommand{\klenght}{\textit{k }}
\newcommand{\slenght}{\textbf{L }}
\newcommand{\vslenght}[1]{\textit{#1}}

Questo progetto tratta della realizzazione di un framework in python per il conteggio dei k-mer e la selezione dei k-mer interessanti per il riconoscimento della filogenesi.

\section{Motivazione}

\section{Scopo del Progetto}
Lo scopo del progetto \`e di costruire un framework in \textbf{python} per il riconoscimento della filogenesi di una sequenza di RNA.
Il progetto, in particolare, cerca di predire il \textbf{Phylum}, la \textbf{Classe}, e l'\textbf{ordine} di una sequenza.
Il framework \`e composto da due parti: la prima parte del framework servir\`a a calcolare i k-mer di una sequenza genetica.
La seconda parte del framework serve per ridurre i k-mew che non sono utili al riconoscimento della filogenesi.
Il riconoscimento della filogenesi avverr\`a tramite i k-mer presenti all'interno della stessa.

\section{Contesto Generale}
    In questa sezione, in primo luogo, verr\`a spiegato quale alfabeto viene preso in considerazione per le sequenze
    di RNA.Successivamente verr\`a spiegato che cosa \`e un k-mer e come si pu\`o estrarre un k-mer da una sequenza di RNA.
    \subsubsection{Alfabeto Accettato}
        In una sequenza di RNA vengono accettati i seguenti caratteri.\\ \\
        \begin{tabular}{|l|c|}
            \hline
            \textbf{Carattere} & \textbf{Significato}\\
            \hline
            ``A'' & Adenosina\\
            \hline
            ``C'' & Citosina\\
            \hline
            ``G'' & Guanina\\
            \hline
            ``U'' & Uracile\\
            \hline
            ``R'' & Adenosina o Guanina\\
            \hline
            ``Y'' & Citosina o Uracile\\
            \hline
            ``S'' & Guanina o Citosina\\
            \hline
            ``W'' & Adenosina o Uracile\\
            \hline
            ``K'' & Guanina o Uracile\\
            \hline
            ``M'' & Adenosina o Citosina\\
            \hline
            ``B'' & Citosina o Guanina o Uracile\\
            \hline
            ``D'' & Adenosina o Guanina o Uracile\\
            \hline
            ``H'' & Adenosina o Citosina o Uracile\\
            \hline
            ``V'' & Adenosina o Citosina o Guanina\\
            \hline
            ``N'' & Qualsiasi base\\
            \hline
            ``.'' o ``-'' & gap\\
            \hline
        \end{tabular}

    \subsubsection{Definizione ed esempi di k-mer e della loro frequenza}
    Data una sequenza biologica, un k-mer \`e una sottostringa della medesima, di lunghezza \klenght.
    Quando \klenght \`e di lunghezza uguale ad 1, i k-mer vengono chiamati monomeri.
    Il conteggio della frequenza dei k-mer viene utilizzato nella bioinformatica per analizzare le sequenze di DNA/RNA.
    Nel nostro caso vengono analizzati i k-mer delle sequenze di RNA per ricostruire la filogenesi della sequenza
    tramite l'utilizza di un modello di machine learning.
    Per filogenesi si intende il processo di ramificazione delle linee di discendenza.
    Il conteggio dei k-mer pu\`o essere utilizzato come firma della sequenza analizzata. Infatti
    il numero di k-mer presenti in una sequenza di lunghezza \slenght \`e uguale a
    \begin{equation*}
        L-k+1
    \end{equation*}
    \textbf{Esempio dei k-mer all'interno di una sequenza di RNA:}
    \\
    Prendiamo come esempio la sequenza:``ACGACUYN''.
    Mentre come valore \klenght prendiamo uno, inizialmente.
    La sequenza \`e di lunghezza \vslenght{8}, di conseguenza il numero dei k-mer presenti nella sequenza \`e di \text{8}.
    Il primo k-mer individuato \`e il k-mer ``A'' (il primo carattere all'interno della sequenza).
    Il k-mer ``A'' avr\`a frequenza due, visto che \`e presente due volte all'interno della sequenza.
    Il prossimo k-mer di lunghezza uno presente \`e ``C'', con frequenza due.
    Ecco l'elenco di tutti i k-mer con la loro frequenza di lunghezza 1:
    \begin{itemize}
        \item A:2
        \item C:2
        \item G:1
        \item U:1
        \item Y:1
        \item N:1
    \end{itemize}
    Successivamente si imposter\`a il valore \klenght a due.
    Di conseguenza il numero di k-mer con \klenght uguale a due \`e di sette.
    Per \klenght uguale a due il primo k-mer disponibile \`e ``AC'', con frequenza due.
    Il secondo k-mer della sequenza sar\`a ``CG'' che ha frequenza uno.
    Di seguito l'elenco con la frequenza di tutti i k-mer di lunghezza due della sequenza ``ACGACUYN'':
    \begin{itemize}
        \item AC:2
        \item CG:1
        \item GA:1
        \item CU:1
        \item UY:1
        \item YN:1
    \end{itemize}
    E cos\`{\i} via fino al valore \klenght desiderato.