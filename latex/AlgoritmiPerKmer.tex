\chapter{Algoritmi Per Kmer}

\label{ch:algoritmiperkmer}
In questo capitolo si andr\`a a discutere degli algoritmi utilizzati per il calcolo dei k-mer.
Gli algoritmi che abbiamo implementato sono presenti nell'articolo scientifico presente nella bibliografia~\cite{algoritmiK-mer}.
Secondo lo studio, i migliori algoritmi per il calcolo dei k-mer sono \textbf{DSK}, \textbf{Gerbil} e \textbf{KMC3}.
Tutti e tre gli algoritmi sono detti ``disk based'', ovvero fanno affidamento sul disco fisso per cercare un buon compromesso
tra l'utilizzo della memoria principale utilizzata e tempo di esecuzione.
Di seguito verrano spiegati il loro funzionamento e la loro implementazione.

\section{DSK}\label{sec:dsk}
L'algoritmo DSK(disk streaming k-mer)~\cite{DSK} \`e un algoritmo per il calcolo dei k-mer
in cui viene decisa dall'utente la quantità di memoria e spazio sul disco
questo approccio attua un compromesso tra memoria, disco e tempo.
Il multiset di tutti i k-mer viene partizionato e ogni partizione viene salvata
nel disco poi, ogni partizione viene caricata separatamente in memoria in una
hash table temporanea.
Il conto dei k-mer esce fuori ripercorrendo ogni hash table.
L'algoritmo prevede una prima fase in cui vengono calcolati:
\begin{itemize}
 \item il numero di k-mer presenti in una sequenza.
 \item il numero d'iterazioni necessarie a leggere, salvare su disco e contare tutti i k-mer e la loro frequenza nella sequenza.
 \item Il numero di partizioni necessarie per ogni iterazione.
\end{itemize}
Ad ogni iterazione vengono letti tutti i k-mer e \textbf{se \`e l'iterazione giusta}, il k-mer letto viene salvato in una delle partizioni create.
La formula utilizzata per decidere se salvare o no il k-mer in una partizione \`e se il k-mer \textit{m } rispetta il seguente predicato:
\begin{equation}\label{eq:predicatokmer}
    (h(m) \mod N_{iters}) == i
\end{equation}
dove:
\begin{itemize}
    \item m \`e il kmer letto.
    \item h(m) \`e una funzione di hash applicata al k-mer m.
    \item Niters \`e il numero totale d'iterazioni.
    \item i \`e l'iterazione attuale.
\end{itemize}
Se il k-mer rispetta il predicato soprastante (\ref{eq:predicatokmer}) allora verr\`a salvato nella partizione j.
La formula per determinare in quale partizione salvare il k-mer \`e la seguente:
\begin{equation} \label{eq:partizionegiusta}
    j = (h(m)/N_{iters}) \mod N_{p}
\end{equation}
dove:
\begin{itemize}
    \item j \`e la partizione attuale.
    \item m \`e il k-mer attuale.
    \item Niters \`e il numero d'iterazioni totali.
    \item h(m) \`e la funzione di hash applicata al k-mer m.
    \item Np \`e il numero totale di partizioni necessarie.
\end{itemize}
\subsubsection{Calcolo del numero di iterazioni}%\label{subsec:calcolo-del-numero-di-iterazioni}
Il numero d'iterazioni \`e dato dalla sequente formula:
\begin{equation*}
 N_{iters} = \lceil{\frac{v*2^{\log_2{2k}}}{D}}\rceil
\end{equation*}
Dove:
\begin{itemize}
 \item v \`e il numero di k-mer in una sequenza.
 \item k \`e la dimensione dei k-mer.
 \item D \`e dimensione su disco occupata dalle partizioni.
\end{itemize}
\subsubsection{Calcolo del numero di Partizioni necessarie ad ogni Iterazione:}%\label{subsec:calcolo-del-numero-di-partizioni-necessarie-ad-ogni-iterazione:}
Il numero di partizioni necessarie ad ogni iterazione \`e dato da:
\begin{equation*}
 N_{p} = \lceil{\frac{v*(2^{\lceil\log_2{2k}\rceil}+32)}{0.7*N_{iters}*M}}\rceil
\end{equation*}
Dove:
\begin{itemize}
 \item v \`e il numero di k-mer presenti nella sequenza.
 \item k \`e la dimensione dei k-mer.
 \item N\_{iters} \`e il numero d'iterazioni necessarie.
 \item M \`e la dimensione in memoria che l'algoritmo utilizzer\`a ad ogni iterazione.
\end{itemize}
Si noti che a ogni iterazione non \`e detto che vengano utilizzate tutte le partizioni per salvare i k-mer.
\subsection{Descrizione del codice utilizzato per l'implementazione:}
Di seguito verr\`a mostrata l'implementazione della lettura e del salvataggio nelle partizioni dell'algoritmo \textbf{DSK}.
\begin{lstlisting}[label={lst:parte1}]
 while kmer_reader.has_next(k_number):
   kmer = kmer_reader.read_next_kmer()
   dsk_utils = DefaultDSKUtils(j, kmer)
   dsk_utils.set_partition_number(partition_number)
   dsk_utils.set_iteration_number(iteration_number)
   if dsk_utils.equals_to_ith_iteration():
     dsk_utils.set_partition_index()
     path = os.path.join(self.__partition_path,filename.split(".")[0])
     path = os.path.join(path, "partition-"+
                   str(dsk_utils.get_partition_index())+".bin")
     dsk_utils.write_to_partitions(path,kmer)
\end{lstlisting}
La classe ``DefaultDSKUtils'' implementa le formule~\eqref{eq:predicatokmer} e~\eqref{eq:partizionegiusta} e si occupa di salvare nelle partizioni il k-mer.
Infatti a ogni k-mer letto, l'oggetto creato dalla classe ``DefaultDSKUtils'' gli viene passato il kmer letto e il numero dell'iterazione attuale.
Poi all'oggetto creato viene assegnato il numero totale d'iterazioni e di partizioni create.
Il predicato~\eqref{eq:predicatokmer} viene implementato dal metodo ``equals\_to\_ith\_iteration'', mentre la formula~\eqref{eq:partizionegiusta} \`e implementata dal metodo ``set\_partition\_index''.
La funzione di hash utilizzata \`e \textbf{MD5}.
Una volta che i k-mer sono salvati nelle partizioni verranno letti e poi salvati nel file di uscita:
\begin{lstlisting}[label={lst:codedsk2ndpart}]
    for j in range(partition_number):
      hash_table = self.initialize_dict()
      path = os.path.join(self.__partition_path, filename)
      path = os.path.join(path, "partition-" + str(j) + ".bin")
      partition_kmer_reader = PartitionKmerReader(path, self.__k)
      size = partition_kmer_reader.get_file_lenght()
      while partition_kmer_reader.has_next(size):
        m = partition_kmer_reader.read_next_kmer()
        s = m.decode("utf-8")
        if s in hash_table:
          hash_table[s] = hash_table[s] + 1
        else:
          hash_table[s] = 1
      if os.path.exists(path):
        os.remove(path)
        out_writer = OutputWriter(filename=molecule_name
                                  , path=self.__out_path)
        hash_table = self.__sort_dictionary(hash_table)
        out_writer.write_to_output(hash_table)
        out_writer.close_all_files()
\end{lstlisting}
Nel codice soprastante si leggono man mano i k-mer e si inseriscono temporaneamente in un dizionario la cui chiave corrisponde a un k-mer e il valore la sua frequenza.
\section{Gerbil}\label{sec:gerbil}
Gerbil~\cite{gerbil} ,
\`e un algoritmo che legge i k-mer, li salva nelle partizioni, poi li conta e salva nel file di uscita.
Gerbil per risparmiare lo spazio occupato dalle partizioni utilizza i minimizer.
Gerbil \`e un algoritmo che lavora in modo pi\`u efficiente quando la lunghezza dei k-mer \`e maggiore o uguale di 32.
\`E un algoritmo che \`e diviso in pi\`u fasi:
\begin{enumerate}
    \item viene letto un k-mer.Dal k-mer letto viene estratto il super k-mer.il super k-mer viene salvato in una partizione.
    \item Vengono letti i super k-mer dalle partizioni, vengono trasformati in k-mer, vengono contati e salvati nel file di output.
\end{enumerate}
\subsection{Minimizers e Super k-mer}\label{subsec:minimizers-e-super-k-mer}
I minimizer sono un modo per salvare pi\`u k-mer contigui risparmiando spazio.
Un super k-mer \`e una sottostringa di un k-mer.
\`E infatti grazie al super k-mer che i minimizer funzionano.
\subsubsection{Esempio:}
Si consideri la sequenza ``ACGACCUNNACCC'' e il valore \textit{k }uguale a 4.
La sequenza \`e composta dai seguenti k-mer:
ACGA,CGAC,GACC,ACCU,CCUN,CUNN,UNNA,\\
NNAC,NACC,ACCC.
Nei k-mer contigui ci sono alcune parti ripetute.
Ad esempio i k-mer ACGA e CGAC hanno la sottostringa ``CGA'' in comune.
Per estrarre i super k-mer da un k-mer bisogna prima decidere prima la lunghezza del minimizer.
Il minimizer determina quale quanti nucleotidi dei k-mer contigui verranno eliminati.
I super k-mer di lunghezza 2 estratti dalla sequenza con \textit{k }uguale a quattro sono:\\ \\
\begin{tabular}{|l|c|}
    \hline
    \textbf{minimizer} & \textbf{super k-mer}\\
    \hline
    GA & AA,CC,CC\\
    \hline
    CU & AC,CN,NN\\
    \hline
    NA & UN,NC,CC\\
    \hline
    CC & AC\\
    \hline
\end{tabular}
\\ \\
Se \textit{k } \`e la lunghezza del k-mer e \textit{m } \`e la lunghezza del minimizer, allora il numero massimo di super k-mer contigui che quel minimizer pu\`o memorizzare \`e uguale a:
\begin{equation} \label{eq:skmersize}
    k - l + 1
\end{equation}
\subsection{Implementazione di Gerbil:} \label{subsec:implementazione-di-gerbil:}
Di seguito l'implementazione della prima parte di Gerbil:
\begin{lstlisting}[label={lst:primapartegerbil}]
        minimizer = ""
        gerbil_utils = DefaultMinimizerHandler(self.__k, self.__m)
        kmer_list = list()
        while reader.has_next(size):
            kmer = reader.read_next_kmer()
            if min_ith == 0:
                minimizer = gerbil_utils.get_minimizers_from_kmer(kmer)
                min_ith += 1
                kmer_list.append(kmer)
            elif min_ith == self.__super_kmer_length - 1:
                kmer_list.append(kmer)
                gerbil_utils.find_super_kmer_and_write(kmer_list
                                                    , minimizer,
                                                partition_file_path)
                min_ith = 0
                kmer_list.clear()
            else:
                min_ith += 1
                kmer_list.append(kmer)
        if len(kmer_list) > 0:
            gerbil_utils.find_super_kmer_and_write(kmer_list
                                                    , minimizer
                                                    , partition_file_path)
\end{lstlisting}
La variabile ``minimizer'' \`e una stringa inizialmente vuota, ma che servir\`a come appoggio temporaneo ai minimizer trovati.
L'oggetto ``gerbil\_utils'' serve a gestire i vari minimizer e super k-mer.
Con il metodo ``get\_minimizers\_from\_kmer'' \`e possibile estrarre un minimizer da un k-mer.
Esso restiuisce gli ultimi \textit{m } caratteri da un k-mer.
Successivamente quando si saranno letti il numero dei k-mer ricavati dalla formula~\eqref{eq:skmersize}, questi verranno passati insieme al minimizer al metodo ``find\_super\_kmer\_and\_write'', che si occuper\`a di eliminare dai k-mer letti il minimizer e di creare (se non esiste gi\`a) e salvare nella partizione i super k-mer generati.
Se ad esempio la sequenza \`e uguale ad ``AACACG''.
Se il minimizer \`e ``CA'' (visto che il primo k-mer letto \`e ``AACA''), allora in ``find\_super\_kmer\_and\_write'' verranno creati i super k-mer AA,AC,CG.
I super k-mer saranno scritti nelle partizioni tutti attaccati (quindi nel nostro esempio diventano ``AAACCG''), mentre il minimizer verr\`a usato nel nome della
partizione.
Se la partizione esiste gi\`a vuol dire che precedentemente \`e gi\`a stato trovato un altro minimizer identico.
Di seguito l'implementazione della seconda parte di Gerbil:
\begin{lstlisting}[label={lst:secondapartegerbil}]
        while reader.has_next(size - 1):
            kmer = reader.read_next_kmer()
            if kmer in hash_table:
                d[kmer] = d[kmer] + 1
            else:
                d[kmer] = 1
\end{lstlisting}
Questo pezzo di codice si occupa di leggere i super k-mer dalle partizioni, rimettere i minimizer al loro posto e salvarlo in un dizionario, dove la chiave \`e il k-mer,
mentre il valore la frequenza.
La lettura e il ricomponimento del k-mer \`e effettuato dall'oggetto ``reader'', che appartiene alla classe ``SuperKmerReader''.
L'oggetto conosce il minimizer visto che \`e il nome della partizione e conosce anche quanto \`e lungo il super k-mer.
Lui sa che al primo super k-mer letto, il minimizer deve essere aggiunto alla fine.
Il secondo super k-mer, il minimizer deve essere aggiunto tra il penultimo e l'ultimo carattere.
Il terzo super k-mer, il minimizer deve essere aggiunto tra il terzultimo e il penultimo carattere.
Di seguito implementation del metodo ``read\_next\_kmer'' della classe ``SuperKmerReader'':
\begin{lstlisting}[label={lst:superkmerreader}]
super_kmer = self.__file.read(self.__super_kmer_size)
        super_kmer = super_kmer.decode("utf-8")
        if self.__ith == self.__super_kmer_size:
            kmer = super_kmer + self.__minimizer
        elif self.__ith == 0:
            kmer = self.__minimizer + super_kmer
        else:
            kmer = super_kmer[:self.__ith] + self.__minimizer
                                    + super_kmer[self.__ith:]
        self.__ith -= 1
        if self.__ith < 0:
            self.__ith = self.__super_kmer_size
        self.__size_counter += 1
        return kmer
\end{lstlisting}
La variabile ``self.\_\_ith'' serve per sapere in quale posizione inserire il minimizer quando si legge il super k-mer.
Inizialmente sar\`a impostata all'ultimo carattere, poi man mano che i super k-mer verranno letti, questa scala fino al primo carattere.
In questo caso se nella partizione ci sono ancora super k-mer la variabile viene impostata di nuovo all'ultimo carattere.
%\subsubsection{Implementazione del minimizer}
%Nel progetto, nella cartella ``Main/kmer/Utils/minimizer'' \`e presente la classe ``DefaultMinimizer'' e la classe ``DefaultMinimizerHandler''.
%Queste classi si occupano di implementare e gestire i minimizers e i super k-mer associati ai minimizer.
%Inizialmente non \`e impostato nessun minimizer, quindi il primo k-mer deve essere passato alla classe tramite il metodo ``get_minimizers_from_kmer'' della classe ``DefaultMinimizerHandler''.
%Questo metodo \`e implementato nel seguente modo:
%\begin{lstlisting}[label={lst:getminfromkmer}]
    %def get_minimizers_from_kmer(self, kmer):
            %self.__minimizer = kmer[-m:]
            %return self.__minimizer
%\end{lstlisting}
%Questo metodo estrae gli ultimi \textit{m }caratteri da un k-mer, li assegna al minimizer e lo restituisce.
%Successivamente il metodo ``''

\section{KMC3}\label{sec:kmc3}
KMC3~\cite{KMC3} \`e un algoritmo per il conteggio della frequenza dei k-mer.
Questo algoritmo \`e molto simile a gerbil. La pi\`u grande differenza \`e che le partizioni, nella seconda fase, vengono letti in base all'ordine lessicografico dei minimizer.
L'algoritmo utilizza, per ordinare i minimizers, l'algoritmo \textbf{``Most Significant Radix Sort'' } (da qui in avanti verr\`a chiamato MS radix Sort).
KMC3 \`e diviso in due fasi:
\begin{enumerate}
    \item viene letto un k-mer.Dal k-mer letto viene estratto il super k-mer.il super k-mer viene salvato in una partizione.
    \item Vengono letti i minimizer delle partizioni, vengono ordinati e vengono letti i super k-mer dalle partizioni nell'ordine trovato.Poi vengono trasformati in k-mer, vengono contati e salvati nel file di output.
\end{enumerate}
\subsection{implementazione della prima fase:}\label{subsec:implementazione-della-prima-fase:}
In questo caso il codice \`e identico alla prima fase di gerbil.
\begin{lstlisting}[label={lst:prima-parte-KMC3}]
    minimizer = ""
        kmer_list = list()
        mh = DefaultMinimizerHandler(self.__k, self.__m)
        while reader.has_next(length):
            kmer = reader.read_next_kmer()
            if min_ith == 0:
                kmer_list.append(kmer)
                minimizer = mh.get_minimizers_from_kmer(kmer)
                min_ith += 1
            elif min_ith == self.__m - 1:
                kmer_list.append(kmer)
                mh.find_super_kmer_and_write(kmer_list, minimizer, partition_sub_path)
                kmer_list.clear()
                min_ith = 0
            else:
                kmer_list.append(kmer)
                min_ith += 1
        if len(kmer_list) > 0:
            mh.find_super_kmer_and_write(kmer_list, minimizer, partition_sub_path)
\end{lstlisting}
Il modo in cui i super k-mer e i minimizers vengono letti e creati \`e lo stesso di Gerbil.
\subsection{implementazione della seconda fase:}\label{subsec:seconda-fase:}
In questa fase le partizioni vengono lette nell'ordine dettato da ``MS radix Sort''.
MS radix sort, come si pu\`o dedurre dal nome, inizia ad ordinare le stringhe a partire dal carattere pi\`u significativo.
L'algoritmo di ordinamento \`e stato implementato nel seguente modo dalla classe ``MostSignificantRadixSort'':
\begin{lstlisting}[label={lst:ms-radix-sort}]
    exp = self.__string_length-1
        i = 0
        while exp > i:
            self.counting_sort(i)
            i += 1
\end{lstlisting}
Come si pu\`o notare l'algoritmo utizza un altro algoritmo di ordinamento per ordinare le stringhe in base al carattere che si sta analizzando, ovvero \textbf{``Counting Sort''}.
Una volta che si ha l'ordine in cui le partizioni devono essere lette, i super k-mer vengono trasformati in kmer dalla classe ``SuperKmerReader'' (descritta nel capitolo \ref{subsec:implementazione-di-gerbil:}. )
Successivamente vengono inseriti in un dizionario in cui la chiave \`e il k-mer, mentre il valore corrisponde alla frequenza di quel k-mer.
Di seguito l'implementazione della seconda fase:
\begin{lstlisting}[label={lst:implementazione-seconda-fase-KMC3}]
    size = reader.get_file_lenght()
            ht = dict()
            while reader.has_next(size - 1):
                kmer = reader.read_next_kmer()
                if kmer in ht:
                    ht[kmer] += 1
                else:
                    ht[kmer] = 1
                if len(ht) > 64:
                    self.__write_kmer_count(part_name, ht)
                    ht.clear()
            if len(ht) > 0:
                self.__write_kmer_count(part_name, ht)
                ht.clear()
\end{lstlisting}
Come si pu\`o notare al massimo verranno letti 64 k-mer alla volta.
Questo \`e stato fatto per non occupare troppa memoria principale.
Infatti una partizione pu\`o contenere molti super k-mer.
\section{Formato del file di uscita}\label{sec:formato-del-file-di-uscita}
Il file di output \`e in formato ``CSV''.
Nell'intestazione sono presenti i k-mer trovati in almeno una le molecole analizzate.
Se ad esempio la molecola A ha il k-mer ``CC'' e la molecola B non lo ha, questo verr\`a comunque inserito nell'header.
Il primo valore nell'intestazione \`e l'id.
L'id corrisponde al nome della molecola.
Dopo l'id, nell'intestazione sono presenti i k-mer messi prima in ordine di lunghezza e poi in ordine alfabetico.
Ad esempio il k-mer ``U'' verr\`a inserito prima del k-mer ``AA''.
\subsection{Esempio del file di uscita:}\label{subsec:esempio-del-file-di-uscita:}
Prendiamo per esempio la molecola di nome ``A'' e la molecola di nome ``B''.
La molecola A possiede i seguenti k-mer:
\begin{itemize}
    \item C con frequenza 2
    \item CA con frequenza 3
    \item UU con frequenza 1
\end{itemize}
La molecola B possiede i seguenti k-mer:
\begin{itemize}
    \item U con frequenza 10
    \item GC con frequenza 2
    \item UU con frequenza 3
\end{itemize}
Allora il file di uscita sar\`a fatto nel seguente modo:
\\
id,C,U,CA,CG,UU\\
A,2,0,3,0,1\\
B,0,10,0,2,3\\
