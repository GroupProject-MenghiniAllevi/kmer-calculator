\chapter{Selezione Delle Features}
\label{ch:selezionedellefeatures}

In questo capitolo andremo a discutere della selezione delle features.
Nel nostro caso le features sono i k-mer.
L'obiettivo della riduzione delle features \`e la diminuzione delle features inutili per
determinare la filogenesi delle molecole.
Per selezionare le features, nella nostra implementazione abbiamo utilizzato la libreria \textbf{Scikit }.
\section{Bassa varianza}\label{sec:low-variance}
Con questo metodo \`e possibile ridurre le features la cui varianza \`e minore dell'80\%.
\subsection{implementazione della riduzione delle features a bassa varianza:}\label{subsec:implementazione-della-riduzione-delle-features-a-bassa-varianza:}
L'implementazione \`e la seguente:
\begin{lstlisting}[label={lst:low-variance}]
        def apply_low_variance(self, threshold):
            sel = VarianceThreshold(threshold=threshold)
            arr = sel.fit_transform(X=self.__df)
            self.__selected_features = sel.get_support()
            self.__output_arr = arr
\end{lstlisting}
Il parametro ``threshold'' corrisponde al valore minimo della varianza per cui una feature \`e selezionata.
\section{LinearSVC con Penalit\`a L1}\label{sec:linearsvc-penalita-l1}
Questo metodo~\cite{l1-based} utilizza un modello lineare chiamato ``LinearSVC'' con una penalit\`a al modello chiamata L1.
LinearSVC sta per ``Linear Support Vector Classification''.
Un modello ``LinearSVC'' \`e un metodo simile a SVC, ma implementato con una libreria diversa, che permette pi\`u flessibilit\`a
nella scelta delle penalit\`a e della funzione di loss.
Nel nostro caso \`e stato implementato nel seguente modo:
\begin{lstlisting}[label={lst:l1-based}]
    y = np.asarray(list(self.__molecule_expected.values()))
        lsvc = LinearSVC(C=0.1, penalty="l1", dual=False).fit(X=self.__df, y=y)
        model = SelectFromModel(lsvc, prefit=True)
        self.__output_arr = model.transform(X=self.__df)
\end{lstlisting}
\section{Alberi di ricorsione}\label{sec:alberi-di-ricorsione}
Questo metodo permette di rimuovere le features inutili tramite un classificatore chiamato ``ExtraTreeClassificator''.
Il classificatore si adatta a un numero di alberi decisionali randomizzati su vari sotto campioni del dataset e utilizza la media
per migliorare l'accuratezza predittiva e il controllo dell'overfitting.
\begin{lstlisting}[label={lst:tree}]
    y = np.asarray(list(self.__molecule_expected.values()))
        clf = ExtraTreesClassifier(n_estimators=5)
        clf = clf.fit(self.__df, y)
        model = SelectFromModel(clf, prefit=True)
        self.__output_arr = model.transform(self.__df)
\end{lstlisting}

\section{Test del chi2}\label{sec:test-del-chi2}
Il test del chi quadrato \`e un test statistico.
Esso trasforma i valori in valori positivi facendone il quadrato.
Chi quadrato misura la dipendenza tra variabili stocastiche, quindi questo metodo elimina le features
che hanno maggiori probabilit\`a di essere indipendenti dalla classe.
\begin{lstlisting}

\end{lstlisting}


